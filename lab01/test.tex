\documentclass[11pt,oneside]{article}
\usepackage{geometry}			% See geometry.pdf to learn the layout options.
								% There are lots.
								
\geometry{letterpaper}			% ... or a4paper or a5paper or ... 
%\geometry{landscape}			% Activate for rotated page geometry
%\usepackage[parfill]{parskip}	% Activate to begin paragraphs with an empty
								% line rather than an indent

\usepackage{ifpdf}
\usepackage{graphicx}
\usepackage{booktabs}
\usepackage[utf8]{inputenc}

% Definitions
\def\myauthor{Author}
\def\mytitle{Title}
\def\mycopyright{\myauthor}
\def\mykeywords{}
\def\mybibliostyle{plain}
\def\mybibliocommand{}
\def\mysubtitle{}



%
%	PDF Stuff
%

\ifpdf
  \pdfoutput=1
  \usepackage[
  	plainpages=false,
  	pdfpagelabels,
  	bookmarksnumbered,
  	pdftitle={\mytitle},
  	pagebackref,
  	pdfauthor={\myauthor},
  	pdfkeywords={\mykeywords}
  	]{hyperref}
  \usepackage{memhfixc}
\fi


% Title Information
\title{\mytitle \\ \mysubtitle}
\author{\myauthor}

\begin{document}


% Title Page

\maketitle

% Copyright Page
\textcopyright{} \mycopyright

%
% Main Content
%


% Layout settings
\setlength{\parindent}{1em}

\section{\textbf{COMP2911} Engineering Design in Computing}
\label{comp2911engineeringdesignincomputing}

\subsection{Tutorial - Week 1}
\label{tutorial-week1}

\textbf{Discussion Points}
\label{discussionpoints}

\vskip 2em
\hrule height 0.4pt
\vskip 2em

{\itshape The Good, the Bad and the Ugly (aka Getting to Know You)}
\label{thegoodthebadandtheuglyakagettingtoknowyou}

\begin{enumerate}

\item \textbf{The Ugly:}\begin{itemize}

\item \textbf{Name:} Louis Tiao
\item \textbf{Nickname:} None
\item \textbf{UNSW Program:} Computer Science/Mathematics
\item \textbf{Other:}\ldots
\end{itemize}


\item \textbf{The Bad:}\begin{itemize}

\item \ldots
\end{itemize}


\item \textbf{The Good:}\begin{itemize}

\item \ldots
\end{itemize}


\end{enumerate}

\vskip 2em
\hrule height 0.4pt
\vskip 2em

{\itshape People As Programmers}
\label{peopleasprogrammers}

\begin{enumerate}

\item I see a little bit of myself in most of the programmer predilections from A-Z with respect to the technical inclinations and not so much the personal idiosyncrasies. Specifically, I'm similar to Belligerent Brian and possibly Zealous Zack when it comes to my propensity to ``select cutting edge technologies simply for its buzzword compliance, betting that cool acronyms and shiny new methodologies will make me appear progressive and forward-looking.'' I can also be a bit of a Feature Creep Frank when I indulge in personal projects in that I simply fail to exercise any form of self-restrain from adding a myriad of cool but often frivolous features.
\item Computer Programmers feel that coding is the entirety of their job. It can be said that they are slightly narrow-minded in that they work to close bug tracker issues, while Software Developers have a broad view and deep understanding of the business goals and works to deliver business value and uses technology as a means to an end to solving business problems. It is unlikely that the proverbial ``person-in-the-street.'' {\itshape Their intent is to solve a business problem, not just to close an issue in a bug tracking system.}
\end{enumerate}

\vskip 2em
\hrule height 0.4pt
\vskip 2em

{\itshape Software Requirements}
\label{softwarerequirements}

\begin{enumerate}

\item \textbf{What happened to the Mars Polar Lander in 1999?}

The software triggered the shutdown of the Lander's descent engines, believing that the Lander had already landed on the surface of Mars.


\item \textbf{What went wrong?}

The essential problem was a breakdown in communication of software requirements. The revised requirements never made it into the lower-level requirements. The testing process was also lax in that full regression testing was not performed, so after certain changes were made and bug fixes were done, only a part of the overall system was tested and the touchdown sensor bugs were not detected as a result. 


\item \textbf{What lessons can you learn about team-based software development from this?}

The importance of:

\begin{itemize}

\item Communication
\item Requirement Tracking
\item Regression Testing
\end{itemize}


\end{enumerate}

\vskip 2em
\hrule height 0.4pt
\vskip 2em

{\itshape10 Elements of Good Software Design}
\label{0elementsofgoodsoftwaredesign}

\begin{enumerate}

\item TBH, I feel like most of the elements listed were slightly esoteric at this stage, only understandable by seasoned Object-Oriented developers\ldots But otherwise, everything else was pretty much common-sense and hard not to agree with, e.g.
\begin{itemize}

\item Provides The Necessary Functionality
\item Is As Simple As Current And Foreseeable Constraints Will Allow
\item Eliminates Duplication
\item Is Robustly Documented
\end{itemize}


\item Yes. I feel the order of important is reasonable, given the important of the thing being designed to meet its purpose, and also being as simple as possible.
\item The above points are consistent with these points from Dieter Rams:

\begin{itemize}

\item Good design makes a product useful

{\itshape Synonymous with ``Provides The Necessary Functionality''.}


\item Good design is as little design as possible

{\itshape Consistent with being as simple as current and foreseeable constraints will allow.}


\item Good design is long-lasting

{\itshape A robustly documented product should make the design long-lasting.}


\item Good design makes a product understandable

{\itshape Simplicity, and robust documentation should make the product understandable.}


\end{itemize}


\end{enumerate}

% Bibliography
\bibliographystyle{\mybibliostyle}
\mybibliocommand

\end{document}
